\documentclass[12pt]{article}

\usepackage{amssymb,amsmath,amsthm}
\usepackage{pgfplotstable,booktabs,longtable}
\usepackage{graphicx} % Package for including figures
%\usepackage{psfrag,color}

\title{Homework 2 \protect\\ (Math/CS 471)}

\author{Elijah Perez \protect\\ Natalie Chang}


\date{\vfill\\ 9/13/2020}   



\begin{document}
	\maketitle
	\pagebreak
	\section{Introduction}
     This report shows the convergence rate for Newton's Method across three different functions. This report will outline the basic method used, the raw results, and conclusions drawn.
     
    
     
	
	\section{Method}
	The method used to calculate these results was Newton's Method. This is an iterative method that will approximate the root of any given function. 
	Newtons method is:
	\begin{equation}
		f(x_{n+1}) = x_n - \frac{f(x_n)}{f'(x_n)} 
	\end{equation}
	
	The three functions used to test Newton's method are listed below.

	
	\begin{equation}
		f(x) = x, \label{eq:1} 
	\end{equation}
	\begin{equation}
		f(x) = x^2, \label{eq:2} 
	\end{equation}
	\begin{equation}
		f(x) = sin(x) + cos(x^2). \label{eq:3} 
	\end{equation}

	The results are presented in Section 3, Results. These results use the raw output generated by running Newton's method, and performing calculations to generate the corresponding error and convergence ratios. The stopping criteria for the method used is an error tolerance of $10^{-15}$. All calculations were computed using of Fortran code and Perl script.
	

	

	
	\section{Results}
	Below is a table showing the results of applying Newton's method to the three functions.
	\pgfplotstableset{
		begin table=\begin{longtable},
			end table=\end{longtable},
	}
	\pgfplotstabletypeset[
	col sep=comma,
	string type,
	columns/f/.style={column name=f, column type={c}},
	columns/Error/.style={column name={${(E_{abs})_{n+1}}$}, column type={|l|}},
	columns/Linear/.style={column name={$\frac{(E_{abs})_{n+1}}{(E_{abs})_{n}}$}, column type={l|}},
	columns/Quadratic/.style={column name={$\frac{(E_{abs})_{n+1}}{{(E_{abs})_{n}}^2}$}, column type={l|}},
	every head row/.style={
		before row={\caption{Convergence of Newton's Method for 3 different functions}\\\toprule},
		after row=\midrule,
	},
	every last row/.style={after row=\bottomrule},
	every first column/.style={
		column type/.add={|}{}
	},
	string replace*={function}{\bottomrule\\\\ \bottomrule f(x) \\ Iteration &  {${(E_{abs})_{n+1}}$} & {$\frac{(E_{abs})_{n+1}}{(E_{abs})_{n}}$} & {$\frac{(E_{abs})_{n+1}}{{(E_{abs})_{n}}^2}$} \\\bottomrule},
	]{newton_out.csv}
%	\begin{table}[h]
%		\begin{centering}
%			\begin{tabular}{| l | c |}
%				\hline 
%				\input{Something.tex}  
%				\hline
%			\end{tabular}
%			\caption{Table of Errors. \label{tab:1}}
%		\end{centering}
%	\end{table}
	
	%\bibliographystyle{plain}
	%\bibliography{bibfile}
	
	\section{Conclusion}
	When Newton's method (1) is applied to the function (2) we see rapid convergence to the zero. Because this converged after a single iteration, we cannot classify the convergence rate to be linear or quadratic when applied to (2). When applying (1) to (3) we see a clear linear convergence of Newton's method. This is because the linear convergence ratio $\frac{(E_{abs})_{n+1}}{(E_{abs})_{n}}$ converges to a constant (in this case 0.5). This shows us that the next error depends linearly on the current error. Newton's method has the ability to converge quadratically, so it is evident that the method is performing less than adequately. This is because the true root of (3) is at $x = 0$. Since $f'(0) = 0$, we have  an issue with Newton's method because it contains the term $\frac{f(x_n)}{f'(x_n)}$ which, when approaching the root, would be close to $\frac{0}{0}$. An alternative method that could be used to regain quadratic convergence would be modified Newton's method $f(x_{n+1}) = x_n - 2{\frac{f(x_n)}{f'(x_n)}}$. When applying (1) to (4) we can see that the convergence of Newton's method is quadratic. This is because the quadratic convergence ratio $\frac{(E_{abs})_{n+1}}{{(E_{abs})_{n}}^2}$ is closer to remaining constant compared to linear convergence ratio which we see rapidly decreasing, suggesting a quadratic and not linear relationship between the current error and next error.
\end{document}  
